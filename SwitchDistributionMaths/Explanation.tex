\documentclass[11pt, a4paper]{article}
\title{Switch Distribution}
\author{Felix Benning}

\usepackage{mathtools}

%%%%%%%%%%%%%%%%%%%%% ARROWS %%%%%%%%%%%%%%%%%%%%%%%%%%%%%
\makeatletter
%arrow with explanation 
%(https://tex.stackexchange.com/questions/467779/alignment-of-implication-arrows-with-text-on-top)
% % % % % % % % % % % % % % % % % % % % % % % % % % % % % % % % % % % % % % %
\newcommand*{\sm@shstack}[3][lr]{
	\mathrel{% makes the resulting stack a relation class (like =,<,>,...)
		\smashoperator[#1]{\mathop{#2}^{#3}}}
}

\newcommand*{\lx@rrow}[2]{ %uses leftsides smashoperator thus ignoring overlap towards the left
	\sm@shstack[l]{#2}{#1}
}
%use when aligning to the left
\newcommand*{\x@rrow}[2]{%uses no smashoperator, but adds back stretchability of \implies to the result contrary to \stackrel
	\;\mkern-\thickmuskip\mathop{#2}\limits^{#1}\;\mkern-\thickmuskip
}%use in \(\) and as individual operator without aligning necessities


\newcommand*{\lximplies}[1]{\lx@rrow{#1}{\implies}} %%%%% =>
\newcommand*{\lximpliedby}[1]{\lx@rrow{#1}{\impliedby}} %%%%%% <=
\newcommand*{\lxiff}[1]{\lx@rrow{#1}{\iff}} %%%%%%%%%%%%% <=>
\newcommand*{\lxeq}[1]{\lx@rrow{#1}{=}} %%%%%%%%%% =
\newcommand*{\lxlq}[1]{\lx@rrow{#1}{\le}}
\newcommand*{\lxgq}[1]{\lx@rrow{#1}{\ge}}

\newcommand*{\ximplies}[1]{\x@rrow{#1}{\implies}} %%%%% =>
\newcommand*{\ximpliedby}[1]{\x@rrow{#1}{\impliedby}} %%%%%% <=
\newcommand*{\xiff}[1]{\x@rrow{#1}{\iff}} %%%%%%%%%%%%% <=>
\newcommand*{\xeq}[1]{\x@rrow{#1}{=}} %%%%%%% =
\newcommand*{\xlq}[1]{\x@rrow{#1}{\le}} %%%%%%%

\makeatother
%%%%%%%%%%%%%%%%%%%%%%%%%%%%%%%%%%%%%%%%%%%%%%%%%%%%%%%

\begin{document}
    \maketitle

    \section{Formulate the problem}
    Our distribution should start as a linear function \(f\) until time \(t_0\) and continue as an exponential function \(g\) until the end of time. The total budget shall be \(B\). Then the functions
    \begin{align*}
        f(x) & = a x+1\\
        g(x) & = b \exp(cx)
    \end{align*}
    should satisfy:
    \begin{align}
        f(t_0) &= g(t_0) \label{cond 1}\\
        f'(t_0) &= g'(t_0) \label{cond 2}\\
        B &= \int_0^{t_0} f(x) dx + \int_{t_0}^\infty g(x)dx \label{cond 3}
    \end{align}
    \section{Solve for a,b and c}
    Equation (\ref{cond 1}) implies:
    \begin{align}
        at_0 +1 = b \exp(c t_0) \label{res 1}
    \end{align}
    Equation (\ref{cond 2}) implies:
    \begin{align}
        &a = b c \exp(c t_0) \nonumber\\
        \implies & \frac{a}{c} = b \exp(c t_0) \stackrel{(\ref{res 1})}{=} at_0 +1 \label{res 2}\\
        \implies & a = cat_0 + c \nonumber\\
        \implies & a - act_0 = c \nonumber\\
        \implies & a(1-c t_0) =c \nonumber\\
        \implies & a = \frac{c}{1-ct_0} \label{res 3}
    \end{align}
    Note that \(1-ct_0\ge1>0\), since \(c<0\) will be a necessary assumption for the integral to be finite (c.f. (\ref{c<0})), and \(t_0\ge 0\) simply means that the switch happens in the future.

    \noindent (\ref{res 3}) in (\ref{res 2}):
    \begin{align}
        &b\exp(c t_0)\stackrel{(\ref{res 2})}=\frac{a}{c}\stackrel{(\ref{res 3})}{=} \frac{c}{c(1-ct_0)}\nonumber\\
        \implies & b = \frac{\exp(-ct_0)}{1-ct_0} \label{res 4}
    \end{align}
    So we only have one unknown variable left, which is c. Equation (\ref{cond 3}) implies:
    \begin{align}
        B &= \int_0^{t_0} a x+1 dx + \int_{t_0}^\infty b\exp(cx)dx 
        \nonumber\\
        &= \left[\frac{a}{2}x^2 +x\right]_0^{t_0} + \left[\frac{b}{c} \exp(cx)\right]_{t_0}^\infty 
        \nonumber\\
        &=\frac{a}{2}{t_0}^2 +t_0- \frac{b}{c}\exp(c t_0)
        \label{c<0}\\
        &\lxeq{(\ref{res 4})} \frac{a}{2}{t_0}^2 +t_0-\frac{1}{c(1-ct_0)}
        \nonumber \\
        &\lxeq{(\ref{res 3})} \frac{c}{2(1-ct_0)}{t_0}^2 +t_0-\frac{1}{c(1-ct_0)} \nonumber
    \end{align}
    This implies:
    \begin{align*}
        &c(1-c t_0) B = c^2 \frac{{t_0}^2}{2} + c(1-c t_0)t_0 -1\\
        \implies & cB - c^2t_0B = c^2\frac{{t_0}^2}{2} + ct_0 - c^2t_0^2 -1\\
        \implies & c^2 \underbracket{\left(\frac{{t_0}^2}{2} - t_0^2 + t_0B\right)}_{= t_0 B - \frac{{t_0}^2}{2}} 
        + c\left(t_0 -B \right) -1 = 0
    \end{align*}
    Which has solutions:
    \begin{align}
        c_{1/2} = \frac{B-t_0 \pm \sqrt{(B-t_0)^2 + 4 \left(t_0 B - \frac{{t_0}^2}{2}\right)}}{2 \left(t_0 B - \frac{{t_0}^2}{2}\right)}
    \end{align}
    As we assumed \(c<0\) in (\ref{c<0}), we need to ensure this condition is satisfied:

\end{document}